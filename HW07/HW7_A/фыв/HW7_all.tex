%This is a LaTeX template for homework assignments
\documentclass{article}
\usepackage[utf8]{inputenc}
\usepackage{amsmath}
\usepackage[russian]{babel}
\usepackage{xcolor}
\usepackage{hyperref}
\definecolor{linkcolor}{HTML}{799B03} % цвет ссылок
\definecolor{urlcolor}{HTML}{799B03} % цвет гиперссылок
\usepackage{graphicx}
\usepackage{ulem}

\hypersetup{pdfstartview=FitH,  linkcolor=linkcolor,urlcolor=urlcolor, colorlinks=true}

\begin{document}

\section*{Задача A} 

\subsection*{Условие} 
Найти значения переменных в формуле 3SAT, при которых наибольшее число
скобок принимают истинное значение.

\subsection*{Доказательство NPH}
\uline{Надо доказать, что $3SAT \leq_p MAX3SAT$.  Допустим мы умеем решать $MAX3SAT$, то есть мы нашли такой набор $X = (x_1, x_2, ..., x_n)$ при котором максимальное число скобок (обозначим $k$) истинно. Этот набор $X$ и будет решением $3SAT$, если $k =$ числу скобок. $k \neq$ числу скобок, то $3SAT$ не разрешима.}

\newpage
\section*{Задача  R} 

\subsection*{Условие} 
Рассмотрим такой алгоритм построения максимальной клики: будем каждый раз
удалять из графа вершину минимальной степени, пока не получим полный граф.
Докажите или опровергните, что такой алгоритм дает решение, не более чем в X
раз отличающееся от оптимального.

\subsection*{Решение} 
\textbf{Утверждение.} Если клика размера $N$, то все входящие в нее вершины имеет степень $= N - 1$. Доказательство очевидно.
\\
\hfill \\
\textbf{Утверждение.} Для любого $n$ можно построить граф, в котором степень хотя бы одной вершины будет $n$, а всех остальных вершин будет $\geq n$, а размер максимальной клики будет 2. \\
\#
\uline{Таким графом будет $n$-мерный куб. Степень каждой вершины в него входящий будет равна $n$, так как хроматическое число $n$-мерного куба равно 2. Если хроматическое число графа равно 2, то в нем по определению не существует клик размера больше 2.}
\#
\\
\hfill \\
\textbf{Утверждение.} Для любого $N$, существует граф, в котором есть клика размера $N$ и используя этот алгоритм будет получаться клика размера 2. 
\#
Возьмем клику $C$ размера $N + 1$. Возьмем граф $G$, в котором существует вершина $v$, степень которой $N$, а степень всех остальных вершин $\geq N$. Соединим $C$ и $G$ любым ребром, главное, чтобы оно не касалось вершины $v$. В получившемся графе минимальная степень вершины $N$ и существует вершина степени $N$, входившая в граф $C$. Допустим алгоритм начнет удаление именно с этой вершины. Затем он будет постоянно удалять все вершины графа $C$, так как при удалении вершины графа $C$, степень всех вершин графа $C$ уменьшиться на 1. Таким образом граф $C$ будет полностью удален и задача сведется к нахождению максимальной клики в графе $G$. А у этого графа максимальная клика имеет размер 2 по определению.
\#

\subsection*{Вывод} Последнее утверждение опровергает существование константы $X$.

\newpage
\section*{Задача  U} 

\subsection*{Условие} 
Предложить $\frac{1}{2}$-приближенный алгоритм для решения следующей задачи. Требуется разбить множество вершин неориентированного графа $G=(V,E)$ на два непересекающихся множества $S$ и $T$ таким образом чтобы число ребер $(u,v): u \in S$ и $v \in T$ было максимально.

\subsection*{Решение} 
Начинаем со случайного разбиения. На каждой итерации алгоритма из одного множества переносим вершину в другое множество, таким образом, что бы решение улучшалось. Как только решение перестает улучшаться, алгоритм останавливается. Во время остановки алгоритма верно, что для каждой вершины половина (или больше) ребер ведут в другое множество. Если бы это было не так, то мы могли бы перенести эту вершину и улучшить решение. Это значит, что как минимум половина ребер ведут из одного множества в другое, а значит это $\frac{1}{2}$-приближенный алгоритм.

Использована информация из: \href{https://ru.wikipedia.org/wiki/Максимальный_разрез_графа#CITEREFMitzenmacher,_Upfal2005}{Ссылка на источник} .

\newpage

z-test для оценки доли: $t = \frac{\overline{p} - p_0}{\sqrt{\sigma^2/n}}, \overline{p} = mean(X)$

Нам известна ширина интервала $w$ и погрешность $\alpha$ : $w = 2t_{\alpha/2}\sqrt{\sigma^2 / n}, t - $ распределение Стьюдента с $n-1$ степенями свободы, но можно это заменить на нормальное распределение с параметрами $p_0$ и $\sigma$.
Тогда $$n = \frac{z_{\alpha/2 * \sigma}}{w}$$
\end{document}








